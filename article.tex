\documentclass[a4paper,onecolumn,twoside,12pt]{article}
\usepackage[english,russian]{babel}

\begin{document}
\thispagestyle{empty}


Using ML For Generating Cryptographic 
functions
(working title)
   



\begin{abstract}
In this paper we unite machine learning 
and cryptography: using ML methods 
trying to solve one of the 
most important problem
in cryptography: to find boolean
function with acceptable cryptological
properties. Neural network for making pseudorandom
function is developed. This function 
used as round function in Feistel Network, which 
 we finally test on NIST test battery.


\end{abstract} 

\tableofcontents

\section{Introduction}

Cryptography is widely used in
information security. 
Everyone is using it in messagers, shops, 
in internet of things and many others 
aspects of life. Many users
encrypt its personal data without even 
knowing what cryptography is. 

There are a lot of basic templates
for cryptographic encryption
functions called schemes - Feisteil network, SP-network, XSL-scheme etc. 
Every scheme has adjustable
set of parameters such as 
plaintext and key size, internal
round functions, number of rounds. 
Best practice in constructing new encryption
algorithm is to choose a good scheme, 
then select perfect parameters of chosen scheme.
For example, algorithm AES is a XSL-scheme 
with choosen internal operations, including s-boxes.

Chosing excellent round functions in another part 
of cryptographic art.
When new algorithm is published,
researchers of whole world trying to find 
 weakness in it.
 Besides, absence of attacks 
 doesn't mean that algorithm is strong. 
 Many people, besides,  prefer  concept of 
 <<nothing up my sleeve numbers>> \cite{NMSN}, 
 which require generating s-boxes only 
 by random choise. Otherwise one can say that you
 try to hide specific propeties of your algorithm.
 
 
 On the other hand, ML methods can be applying as
 approach to find a good 
 
 That's why in this paper we use neuronal network for generating good round function 
 for Feisteil Network - one of the most popular scheme.
 Then we test properties of our algorithm on the NIST test battery.




\section{Problem of generating good cryptographic functions}

\section{Feisteil Network}

\section{Chapter about ML which we use}

\section{Applying methods from previous chapter in cryptography (main chapter)}

\section{Methodology of testing and tests' results}

\section{Thoughts about tests' results}

\section{Conclusion}

\begin{thebibliography}{3}
	\bibitem{NMSN} <<Nothing up my sleeve number>> wiki article
\end{thebibliography}


\end{document}
